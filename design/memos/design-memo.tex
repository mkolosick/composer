\documentclass{article}

\usepackage{hyperref}

\makeatletter
\newcommand{\makeheader}{%
  \noindent
  \begin{tabular}{@{}ll}
    \ifdefined\@recipient
      \textsc{To: } & \@recipient{} \\
    \fi
    \ifdefined\@from
      \textsc{From: } & \@from{} \\
    \fi
    \ifdefined\@subject
      \textsc{Subject: } & \@subject{} \\
    \fi
  \end{tabular}
  \vspace{1\baselineskip}
}
\newcommand{\recipient}[1]{\newcommand{\@recipient}{#1}}
\newcommand{\from}[1]{\newcommand{\@from}{#1}}
\newcommand{\subject}[1]{\newcommand{\@subject}{#1}}

\renewcommand\section{\@startsection {section}{1}{\z@}%
    {\z@}
    {-1em}%
    {\itshape}}
\makeatother

\newcommand{\composer}{\texttt{\#lang composer}}

\setlength{\parindent}{0pt}
\setlength{\parskip}{10pt}

\recipient{Matthias Felleisen}
\from{Jared Gentner and Matthew Kolosick}
\subject{Design Proposal for \composer}

\pagenumbering{gobble}

\begin{document}

\makeheader{}

Western classical music has a long history. Over the centuries, both the
notation system and the conception of musical harmony have evolved. For
beginning music students, the full complexity of this system is overwhelming.
Following in the vein of the Racket Student Languages, we propose \composer{} as
a system for gradually working students up to a complete compositional framework
from fundamental building blocks.

The language will provide two sub-languages, one for teachers and one for
students. The language for teachers will allow them to define the
constraints under which their students must operate. This language will allow
teachers to constrain students harmonically and formally.

The second language will be for students to write their pieces. Here students
will write a score under the constraints provided by their teacher. The student
language will present errors when students diverge from the rules their teachers
have provided.

\section*{Language Specification}

Our specification of grammar and scoping rules is available at
\url{https://github.com/mkolosick/composer/blob/master/design/spec.org}.

\section*{Milestones}

We plan to follow these steps in developing \composer{}.
\begin{enumerate}
  \item We will develop the notation language.
  \item We will implement a harmonic analyzer that handles a single key with no
    non-chord tones.
  \item We will then construct the constraint sub-language that only enforces
    constraints using the previously written analyzer.
  \item We will expand the harmonic analyzer to handle key modulations.
  \item We will then write the analyzer for form constraints integrating this
    analysis into the constraint sub-language.
  \item We will then expand the harmonic analysis to allow non-chord tones.
\end{enumerate}

\end{document}
