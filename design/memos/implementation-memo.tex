\documentclass{article}

\usepackage[bottom=1in,top=1in]{geometry}

\linespread{1.5}

\usepackage{hyperref}

\makeatletter
\newcommand{\makeheader}{%
  \noindent
  \begin{tabular}{@{}ll}
    \ifdefined\@recipient
      \textsc{To: } & \@recipient{} \\
    \fi
    \ifdefined\@from
      \textsc{From: } & \@from{} \\
    \fi
    \ifdefined\@subject
      \textsc{Subject: } & \@subject{} \\
    \fi
  \end{tabular}
  \vspace{1\baselineskip}
}
\newcommand{\recipient}[1]{\newcommand{\@recipient}{#1}}
\newcommand{\from}[1]{\newcommand{\@from}{#1}}
\newcommand{\subject}[1]{\newcommand{\@subject}{#1}}

\renewcommand\section{\@startsection {section}{1}{\z@}%
    {\z@}
    {-1em}%
    {\itshape}}
\makeatother

\newcommand{\composer}{\texttt{\#lang composer}}

\setlength{\parindent}{0pt}
\setlength{\parskip}{10pt}

\recipient{Matthias Felleisen}
\from{Jared Gentner and Matthew Kolosick}
\subject{Implementation Progress Report for \composer}

\pagenumbering{gobble}

\begin{document}

\makeheader{}

In its present state, \composer{} serves as a proof of concept for the idea of a
musical teaching language implemented as a DSL. However, the notation as it
stands is unworkable. We have implemented the majority of the technical details
in our proposal, and in using the language have come across the shortcomings of
the notation. Through the implementation process, we have also discovered
additional constraints that might be relevant in a classroom setting. These
observations will guide our future work.

\composer{} currently covers the majority of the milestones from our design
proposal. We have developed notation and constraint languages as well as
implemented harmonic and formal analyses. Our harmonic analysis does not handle
non-chord tones, however the constraints allow for specifying certain types of
key modulations. The formal analysis makes some simplifying assumptions, but
works with scores that meet those assumptions. Handling non-chord tones will
require significant design work, and we would like to further validate the
usefulness of the language before spending the time.

The most significant challenge for implementing \composer{} is writing the
examples for testing. Writing music as S-expressions is not viable. As an
example, we transcribed an eight measure Bach chorale. The transcription is
nearly 600 characters wide. Of particular note is how musical notations rely on
vertical alignment, an aspect poorly handled by textual media.

Vertical alignment was also notable in the other major implementation challenge:
type inference of chords. The algorithm for this took substantial work, and does
not currently guarantee a specific typing, just that one exists. A single score
could have multiple correct typings. This could lead to poor error messages when
a possible typing does not align with user intent. To account for this, we would
like a way for students to annotate certain chords with their intended chord
symbol. This is a very vertical notion that seems to have no reasonable
notation in an S-expression format.

Moving forward, we would like to expand the notation system to include a visual
score editor. This will make it so that students might actually be able to use
the language. It will also open up the possibility of students annotating their
scores and the better error messages that would allow. In terms of the analysis,
we would like to implement non-chord tone recognition and voice-leading
constraints. Lastly, we would like to present the system to teachers and
students to evaluate its utility.

\end{document}
